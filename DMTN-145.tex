\documentclass[modern]{spie}

% lsstdoc documentation: https://lsst-texmf.lsst.io/lsstdoc.html
\input{meta}

% Package imports go here.

% Local commands go here.
%Need for SPIE
\usepackage[breaklinks,colorlinks,urlcolor=blue,citecolor=blue,linkcolor=blue]{hyperref}
\usepackage{longtable}

\newcommand{\docRef}{DMTN-145}
\newcommand{\docUpstreamLocation}{\url{https://github.com/lsst-dm/dmtn-145}}


\begin{document}
%% DO NOT EDIT THIS FILE. IT IS GENERATED FROM db2authors.py"
%% Regenerate using:
%%    python $LSST_TEXMF_DIR/bin/db2authors.py > authors.tex


\author[0000-0003-4141-6195]{William~O'Mullane}
\authorinfo{LSST Project Office, 950 N.\ Cherry Ave., Tucson, AZ  85719, USA}


\date{\today}

\begin{abstract}
In the last two years the software landscape on Rubin Observatory construction project  has changed dramatically. We are transitioning from siloed software teams to a  more integrated approach. LSST has Labview, C++, Python and Java components which use a variety of build systems and have uneven test harnesses. As we approach deployment on the mountain we would like to converge on build and versioning for all systems. In addition we intend to use  a Foreman, Puppet and Kubernetes based deployment system.   In this talk we will outline the vision for software build and deployment across all Rubin Observatory subsystems and where we currently are with it.
As the Commissioning Execution Plan (LSE-390) says, "The project team shall
deliver all reports documenting the as-built hardware and software including:
drawings, source code, modifications, compliance exceptions, and recommendations
for improvement." As a first step towards the delivery of documents that will describe the system at the
end of construction, we are assembling teams for producing of the order 40 papers
that eventually will be submitted to relevant professional journals. The immediate goal is to accomplish
all the writing that can be done without data analysis before the data
taking begins, and the team becomes much more busy and stressed.

This document provides the template for these papers.
\end{abstract}

\keywords{software, observatory}

\title{Bringing Rubin Observatory software together }
\hypersetup{pdftitle={\@title}, pdfauthor={\@author}, pdfkeywords={\@keywords}}


\section{Introduction}

We take it the reader knows Rubin observatory (formerly LSST) and its goals. For details on the observatory and the science goals please see \cite{2008arXiv0805.2366I}. Because the name has recently changed one may still see LSST where it should say Rubin observatory, especially in reference material.

From 2018 to now  the software landscape on the Rubin Observatory construction project has changed dramatically.  We are transitioning from siloed software teams to a  more integrated approach.  The project has LabView, C++, Python and Java components - these are built in different ways and do not employ the same test harnesses. We are attempting to align all build/testing on Jenkins, although this is challenging for LabView for example. In Data Management all testing is done via pytest, all C++ code is exposed to Python so we may test it all in the python layer.  Since we have a common access layer to all Telescope control components (OpenSplice) we could follow the same approach there.
Next we are approaching deployment on the summit. To date deployments in telescope control have been fairly manual. As we shifted more toward Python, containers became more prevalent. Now most components can be deployed with Docker and Docker compose. The next logical step for them is to move to Kubernetes; this may not be possible for the camera control system but we will try to pursue this as much as possible. Data Management is already deploying the Science Platform using Kubernetes. Though the processing software for data releases is containerized it is not yet utilized in that manner.
Finally the bare metal provisioning is not fully automated - we have successfully experimented with Foreman and Puppet to bring up new blades in a selected manner. Our approach here is to provision to Kubernetes as much as possible but for other specific machines, such as camera control, to at least provision the machine with Puppet to the level needed by the camera control system.
There are strong management/cultural issues in bringing these efforts together. These are prevalent in all large projects and some of these issues will be touched on in the talk also.

\section{Background }\label{sect:back}

Rubin observatory  data management \citep{2015arXiv151207914J} has the most software by lines of code and number of people working on it in the observatory. This code is for the system to produce and allow access to the Rubin Observatory Legacy Survey of Space and Time \cite{LSE-163}.

The tools and development process of the data management team have evolved and become well honed since the official star to of the project tin 2014 \citep{2018SPIE10707E..09J}

The next large team is  Telescope and site  \citep{2014SPIE.9145E..1AG} whose software team are responsible for all the control systems on the summit. They either produce or procure each software system and integrate it with the others.

The Rubin Observatory LSST camera \citep{2010SPIE.7735E..0JK} also has a software control system and a readout system. The control system adhere to the agreed telescope and site interface.

 Bringing up the end but underpinning all of this is the IT group, which itself has two parts: one in tucson and the other in Chile.

 Across all of Rubin Observatory the systems engineering team \citep{2014SPIE.9150E..0MC} put in place guidelines and rules. They also drive the  System verification and validation: \citep{2014SPIE.9150E..0NS},





\appendix
% Remove this when you strart your paper

{\bf Initial paper list added here for reference.}

``Editor'' is a responsible team leader but not necessarily the person who will do most of
the required work, or who will eventually become the first author. Both issues will be
handled by individual teams.

\begin{verbatim}

domain: Telescope & Site
editor: Jeff Barr
title: Overview of the LSST Telescope

domain: Telescope & Site
editor: Sandrine Thomas
title: Performance of the LSST Telescope

domain: Telescope & Site
editor: Lynne Jones
title: The LSST Scheduler Overview and Performance

domain: Telescope & Site
editor: Bo Xin
title: Performance of the LSST Active Optics System

domain: Telescope & Site
editor: Tiago Ribeiro
title: LSST Observing System Software Architecture

domain: Camera
editor: Justin Wolfe
title: LSST Camera Optics

domain: Camera
editor: Chris Stubbs
title: LSST Camera Rafts

domain: Camera
editor: Steve Ritz
title: LSST Camera Cryostat

domain: Camera
editor: Ralph Schindler
title: LSST Camera Refrigeration

domain: Camera
editor: Steve Ritz
title: LSST Camera Body and Mechanisms

domain: Camera
editor: Mark Huffer and Tony Johnson
title: LSST Camera Control System and DAQ

domain: Camera
editor: Tim Bond and Aaron Rodman
title: LSST Camera Integration and Tests

domain: Data Management
editor: Leanne Guy
title: Overview of LSST Data Management

domain: Data Management
editor: Michelle Butler
title: LSST Data Facility

domain: Data Management
editor: Tim Jenness
title: LSST Data Management Software System

domain: Data Management
editor: Jim Bosch
title: LSST Data Release Processing

domain: Data Management
editor: Eric Bellm
title: LSST Prompt Data Products

domain: Data Management
editor: Gregory Dubois-Felsmann
title: LSST Science Platform

domain: Data Management
editor: Simon Krughoff
title: LSST Data Management Quality Assurance and Reliability Engineering

domain: Data Management
editor: Leanne Guy (with likely delegation to new DM V&V Scientist)
title: LSST Data Management System Verification and Validation

domain: Data Management
editor: Mario Juric
title: LSST Moving Object Processing

domain: Data Management
editor: Robert Lupton
title: LSST Calibration Strategy and Pipelines

domain: Calibration
editor: Patrick Ingraham
title:  Performance of the LSST Calibration Systems

domain: Calibration
editor: Patrick Ingraham
title: Atmospheric Properties with the LSST Auxiliary Telescope

domain: EPO
editor: Amanda Bauer
title: Overview of LSST Education and Public Outreach

domain: EPO
editor: Ardis Herrold
title: LSST Formal Education Program

domain: EPO
editor: Amanda Bauer
title: LSST EPO: The User Feedback

domain: Commissioning
editor: Chuck Claver
title: LSST Observatory System Operations Readiness Report

domain: Commissioning
editor: Bo Xin
title: Performance of Delivered LSST System

domain: Commissioning
editor: Chuck Claver
title: Active Optics Performance with LSST Commissiong Camera

domain: Commissioning
editor: Chuck Claver
title: LSST Active Optics Performance with the LSST Science Camera

domain: Commissioning
editor: Brian Stalder
title: Integration, Test and Commissioning Results from LSST Commissiong Camera

domain: Commissioning
editor: Kevin Reil
title: LSST Camera Instrumental Signature Characterization, Calibration and Removal

domain: Commissioning
editor: Patrick Hascal
title: Installation and Performance of the LSST Camera Refrigeration System

domain: Commissioning
editor: Andy Connolly
title: Science Validation of LSST Alert Processing

domain: Commissioning
editor: Keith Bechtol
title: Science Validation of LSST Data Release Processing

domain: Commissioning
editor: Michael Reuter
title: Tracking of LSST System Performance with Continuous Integration Methods

domain: Commissioning
editor: Chuck Claver
title: The LSST Science Platform as a Commissioning Tool

domain: Commissioning
editor: Chuck Claver
title: Commissioning Science Data Quality Analysis Tools, Methods and Procedures

domain: Commissioning
editor: Lynne Jones
title: Performance Verification of the LSST Survey Scheduler


\end{verbatim}

% Include all the relevant bib files.
% https://lsst-texmf.lsst.io/lsstdoc.html#bibliographies
\section{References} \label{sec:bib}
\bibliographystyle{spiebib}
\bibliography{local,lsst,lsst-dm,refs_ads,refs,books}

% Make sure lsst-texmf/bin/generateAcronyms.py is in your path
\section{Acronyms} \label{sec:acronyms}
\addtocounter{table}{-1}
\begin{longtable}{p{0.145\textwidth}p{0.8\textwidth}}\hline
\textbf{Acronym} & \textbf{Description}  \\\hline

DAQ & Data Acquisition System \\\hline
DM & Data Management \\\hline
EPO & Education and Public Outreach \\\hline
LSE & LSST Systems Engineering (Document Handle) \\\hline
LSST & Legacy Survey of Space and Time (formerly Large Synoptic Survey Telescope) \\\hline
\end{longtable}


\end{document}
