\documentclass[modern]{spie}

% lsstdoc documentation: https://lsst-texmf.lsst.io/lsstdoc.html
\input{meta}

% Package imports go here.

% Local commands go here.
%Need for SPIE
\usepackage[breaklinks,colorlinks,urlcolor=blue,citecolor=blue,linkcolor=blue]{hyperref}
\usepackage{longtable}

\newcommand{\docRef}{DMTN-145}
\newcommand{\docUpstreamLocation}{\url{https://github.com/lsst-dm/dmtn-145}}


\begin{document}
%% DO NOT EDIT THIS FILE. IT IS GENERATED FROM db2authors.py"
%% Regenerate using:
%%    python $LSST_TEXMF_DIR/bin/db2authors.py > authors.tex


\author[0000-0003-4141-6195]{William~O'Mullane}
\authorinfo{LSST Project Office, 950 N.\ Cherry Ave., Tucson, AZ  85719, USA}


\date{\today}

\begin{abstract}
In the last two years the software landscape on Rubin Observatory construction project  has changed dramatically. We are transitioning from siloed software teams to a  more integrated approach. LSST has Labview, C++, Python and Java components which use a variety of build systems and have uneven test harnesses. As we approach deployment on the mountain we would like to converge on build and versioning for all systems. In addition we intend to use  a Foreman, Puppet and Kubernetes based deployment system.   In this talk we will outline the vision for software build and deployment across all Rubin Observatory subsystems and where we currently are with it.
As the Commissioning Execution Plan (LSE-390) says, "The project team shall
deliver all reports documenting the as-built hardware and software including:
drawings, source code, modifications, compliance exceptions, and recommendations
for improvement." As a first step towards the delivery of documents that will describe the system at the
end of construction, we are assembling teams for producing of the order 40 papers
that eventually will be submitted to relevant professional journals. The immediate goal is to accomplish
all the writing that can be done without data analysis before the data
taking begins, and the team becomes much more busy and stressed.

This document provides the template for these papers.
\end{abstract}

\keywords{software, observatory}

\title{Bringing Rubin Observatory software together }
\hypersetup{pdftitle={\@title}, pdfauthor={\@author}, pdfkeywords={\@keywords}}


\section{Introduction}


In the last two years, the software landscape on LSST has changed dramatically.  We are transitioning from siloed software teams to a  more integrated approach.  LSST has Labview, C++, Python and Java components - these are built in different ways and do not employ the same test harnesses. We are attempting to align all build/testing on Jenkins, although this is challenging for LabView for example. In Data Management all testing is done via pytest, all C++ code is exposed to Python so we may test it all in the python layer.  Since we have a common access layer to all Telescope control components (OpenSplice) we could follow the same approach there.
Next we are approaching deployment on the summit. To date deployments in telescope control have been fairly manual. As we shifted more toward Python, containers became more prevalent. Now most components can be deployed with Docker and Docker compose. The next logical step for them is to move to Kubernetes; this may not be possible for the camera control system but we will try to pursue this as much as possible. Data Management is already deploying the Science Platform using Kubernetes. Though the processing software for data releases is containerised it is not yet utilized in that manner.
Finally the bare metal provisioning is not fully automated - we have successfully experimented with Foreman and Puppet to bring up new blades in a selected manner. Our approach here is to provision to Kubernetes as much as possible but for other specific machines, such as camera control, to at least provision the machine with Puppet to the level needed by the camera control system.
There are strong management/cultural issues in bringing these efforts together. These are prevalent in all large projects and some of these issues will be touched on in the talk also.

\subsubsection{Data Management}

LSST data management system and the data products are described in:

\begin{itemize}
  \item The LSST Data Management System: \cite{2015arXiv151207914J}
  \item Data Products Definition Document: \cite{LSE-163}
\end{itemize}
 %------------------------------------------------------------------------------


\subsubsection{Camera}

\begin{itemize}
   \item Design and development of the LSST camera: \cite{2010SPIE.7735E..0JK}
\end{itemize}
%------------------------------------------------------------------------------


\subsubsection{Telescope and Site}

\begin{itemize}
   \item Telescope and site overview and status in 2014:  \cite{2014SPIE.9145E..1AG}
\end{itemize}
%------------------------------------------------------------------------------

\subsubsection{System Engineering}

\begin{itemize}
   \item LSST systems engineering: \cite{2014SPIE.9150E..0MC}
   \item System verification and validation: \cite{2014SPIE.9150E..0NS}
\end{itemize}
%




\appendix
% Remove this when you strart your paper
\input{appendix}
% Include all the relevant bib files.
% https://lsst-texmf.lsst.io/lsstdoc.html#bibliographies
\section{References} \label{sec:bib}
\bibliographystyle{spiebib}
\bibliography{local,lsst,lsst-dm,refs_ads,refs,books}

% Make sure lsst-texmf/bin/generateAcronyms.py is in your path
\section{Acronyms} \label{sec:acronyms}
\addtocounter{table}{-1}
\begin{longtable}{p{0.145\textwidth}p{0.8\textwidth}}\hline
\textbf{Acronym} & \textbf{Description}  \\\hline

API & Application Programming Interface \\\hline
CA & Control (or Cost) Account \\\hline
CI & Continuous Integration \\\hline
DM & Data Management \\\hline
IT & Information Technology \\\hline
LSE & LSST Systems Engineering (Document Handle) \\\hline
LSST & Legacy Survey of Space and Time (formerly Large Synoptic Survey Telescope) \\\hline
S3 & (Amazon) Simple Storage Service  \\\hline
SLAC & SLAC National Accelerator Laboratory \\\hline
\end{longtable}


\end{document}
