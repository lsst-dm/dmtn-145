\section{Introduction}

We take it the reader knows Rubin observatory (formerly LSST) and its goals. For details on the observatory and the science goals please see \cite{2008arXiv0805.2366I}. Because the name has recently changed one may still see LSST where it should say Rubin observatory, especially in reference material.

From 2018 to now  the software landscape on the Rubin Observatory construction project has changed dramatically.  We are transitioning from siloed software teams to a  more integrated approach.  The project has LabView, C++, Python and Java components - these are built in different ways and do not employ the same test harnesses. We are attempting to align all build/testing on Jenkins, although this is challenging for LabView for example. In Data Management all testing is done via pytest, all C++ code is exposed to Python so we may test it all in the python layer.  Since we have a common access layer to all Telescope control components (OpenSplice) we could follow the same approach there.
Next we are approaching deployment on the summit. To date deployments in telescope control have been fairly manual. As we shifted more toward Python, containers became more prevalent. Now most components can be deployed with Docker and Docker compose. The next logical step for them is to move to Kubernetes; this may not be possible for the camera control system but we will try to pursue this as much as possible. Data Management is already deploying the Science Platform using Kubernetes. Though the processing software for data releases is containerized it is not yet utilized in that manner.
Finally the bare metal provisioning is not fully automated - we have successfully experimented with Foreman and Puppet to bring up new blades in a selected manner. Our approach here is to provision to Kubernetes as much as possible but for other specific machines, such as camera control, to at least provision the machine with Puppet to the level needed by the camera control system.
There are strong management/cultural issues in bringing these efforts together. These are prevalent in all large projects and some of these issues will be touched on in the talk also.

\section{Background }\label{sect:back}

Rubin observatory  data management \citep{2015arXiv151207914J} has the most software by lines of code and number of people working on it in the observatory. This code is for the system to produce and allow access to the Rubin Observatory Legacy Survey of Space and Time \cite{LSE-163}.
The tools and development process of the data management team have evolved and become well honed since the official star to of the project tin 2014 \citep{2018SPIE10707E..09J}
The next large team is  Telescope and site  \citep{2014SPIE.9145E..1AG} whose software team are responsible for all the control systems on the summit. They either produce or procure each software system and integrate it with the others.
The Rubin Observatory LSST camera \citep{2010SPIE.7735E..0JK} also has a software control system and a readout system. The control system adhere to the agreed telescope and site interface.
 Bringing up the end but underpinning all of this is the IT group, which itself has two parts: one in tucson and the other in Chile.
 Across all of Rubin Observatory the systems engineering team \citep{2014SPIE.9150E..0MC} put in place guidelines and rules. They also drive the  System verification and validation: \citep{2014SPIE.9150E..0NS}.

In 2014 when the project officially started these groups were set up semi independantly, system engineering had some guidelines but these were not necessarily followed. IT was not put in a strong position in the construction site and was under staffed.This redulted in multiple problems. For example Camera and DM investigated deployment technologies comming to differnt conclusions (puppet/chef), this should have been driven by IT but IT was not envisioned as underpinning a large  modern software production.

The team for

